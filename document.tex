% !TeX spellcheck = en_US
\documentclass[a4paper]{article}
\usepackage[utf8]{inputenc}
\usepackage{t1enc}
\usepackage[english]{babel}
\usepackage{lmodern}
\usepackage{url}
\usepackage{graphics}
\usepackage{listings}
\usepackage[a4paper, total={5.4in, 8in}]{geometry}

\hyphenation{UPPAAL}
\sloppy

\begin{document}
	
	\newcommand{\specialcell}[2][c]{%
		\begin{tabular}[#1]{@{}c@{}}#2\end{tabular}}
	
	\newenvironment*{mytable}[3]{
		% #1: caption, #2: cimke, #3: oszlopdef		 
		\begin{table}[htbp]	
			\caption{#1}          
			\label{tab:#2}            
			\center%
			\begin{tabular}{#3}
			}
			{
			\end{tabular}
		\end{table}
	}
	
	\pagestyle{plain}
	
	
	
	% angol környezet beállítása
	\nonfrenchspacing
	\setlength{\parindent}{0em}
	\setlength{\parskip}{0.45em}
	
	\title{Language Learning Application: \\ Requirement Specification \\ \begin{large}Software Architectures Homework \end{large}}
	\author{Bence Graics \and Kristóf Verbőczy}	
	\date{}
	\maketitle
	\section*{Introduction}
	This document presents the requirement specification of the software architectures homework \textsl{Language Learning Application}. First, the assigned task is introduced, which is followed by the detailed description of the task. Next, the technical parameters of the software developed for the particular task is presented. Finally, a use-case diagram is given that depicts the essential use-cases of the software.
	\section{The task}
	The task is to create an application aiding foreign language learning. The foreign words and sentences can be organized into lessons by means of associated meta information. A central server is responsible for the management and storage of data. In addition to a server component, the application comprises of a client component supporting the management of the server as well as the conduction of lessons. The platforms, serving as the basis of the application, can be chosen freely.
	
	\section{Detailed task description}
	
	\section{Requirements}
	\begin{mytable}{Requirements}{}{|c|c|}
		\hline
		id1 & The system shall let somebody enter if she typed in an existing user name and the connected password. \\
		\hline
		id2 & The users without admin privileges shall not delete anything from the database. \\
		\hline
		id3 & The GUI side shall validate its input fields. \\
		\hline
		id4 & The system shall defend against SQL injection. \\
		\hline
		id5 & The database shall store the hash values of the passwords instead of the plain text. \\
		\hline
		id6 & The system lets users create new exercises. \\
		\hline
		id7 & The users with admin privileges may delete exercises from the database. \\
		\hline
		id8 & The business logic shall provide a coaching phase before the exercising phase. \\
		\hline
		id9 & The business logic shall provide exercises from the user's knowledge level or below that. \\
		\hline
		id10 & Server shall be able to serve multiple clients.
		
		\hline
	\end{mytable}

	
	\section{Technical parameters}
	Both client and server are going to be written in Java in order to be usable on all platforms. The GUI is going to be designed with JavaFX. The server will use Oracle Database. To run the server one needs Maven and WildFly (which is a Maven plugin). The client and server are going to communicate through REST calls.
	
	\section{Use-cases}
	\begin{figure}[htbp]
		\center
		\resizebox{160mm}{!} {
			\includegraphics{figures/use-case.pdf}
		}
		\caption{Use-case diagram}
		\label{fig:use-case}
	\end{figure}
%	This tutorial presents the \framework\ from a practical aspect. First, a short summary of the framework functionalities is given which is followed by the presentation of composite system elements as well as the semantics a composite system conforms to. Next, an installation guide setting up the \framework\ is presented. Finally, the functionalities of the framework are demonstrated based on a case-study called MoDeS$^3$.

%	\begin{figure}[htbp]
%		\center
%		\resizebox{135mm}{!}{
%			\includegraphics{fig/gui_window.png}
%		}
%		\caption{The window supporting the verification and back-annotation functionalities of the framework.}
%		\label{fig:gui_window}
%	\end{figure}	
	

\end{document}